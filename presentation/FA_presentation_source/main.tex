% ISS presentation template
%
% Change history:
% 24.06.2010    Jürgen Ruoff        Initial creation
% 01.07.2010    Patrick Häcker      Generalization
% 02.07.2010    Patrick Häcker      Adjustment
% 15.11.2010    Patrick Häcker      Improvements
% 20.05.2011    Patrick Häcker      Add presentation type
% 06.01.2012	P. Hermannstädter 	Adapted to ISS, small mods
% 15.07.2019 	A. Bartler			Minor fixes, 16:9 aspect ratio added 



% Insert your name here
\newcommand{\presenter}{Zheming Yin}
\newcommand{\presentershort}{Zheming Yin}
\newcommand{\presenteremail}{PRESENTERMAIL} 		% can be accessed using \presenteremail

% Insert presentation title here
\newcommand{\presentationtitle}{Ray tracing channel simulation for millimeter wave indoor localization system}
\newcommand{\shortpresentationtitle}{Ray tracing channel simulation for millimeter wave indoor localization system}

% Insert type of presentation here (or comment line), probably one of:
% Mitarbeitervortrag, Bachelor-Arbeit, Master-Arbeit, Bachelor thesis, Master thesis
\newcommand{\presentationtype}{Research thesis}

% Insert presentation date here
\newcommand{\presentationdate}{18.07.2024}

% Define the language by uncommenting. If there is an error, try deleting temporary files (*.log, ...)
%\newcommand{\lang}{ngerman}
\newcommand{\lang}{english}

% Uncomment the following line, if you want to create handouts (setting to false does not work!)
%\newcommand{\handoutmode}{true}

% Define aspect ratio to  16:9 or 4:3 by uncommenting
%\newcommand{\aspectratio}{43}
\newcommand{\aspectratio}{169}


% Load beamer class using LSS style
\input{presentation}

\documentclass{beamer}
\usepackage{setspace}
\usepackage[sorting=none]{biblatex}
\addbibresource{refs.bib}
\usepackage{graphicx}
\usepackage{caption}
\captionsetup[figure]{labelformat=simple, labelsep=period}
\newcounter{section}
\setcounter{section}{0}
\renewcommand{\thefigure}{\thesection.\arabic{figure}}
\usepackage{animate}


% My commands:

% -----------------------------------------------------------------------------
% -----------------------------------------------------------------------------
\begin{document}
\lstset{basicstyle=\small\ttfamily,xleftmargin=18pt,language=Python,
        commentstyle=\color{green},showstringspaces=false,stringstyle=\color{magenta}\ttfamily}

% -----------------------------------------------------------------------------
% This is the title page
\begin{frame}[t,plain]
	\titlepage
\end{frame}


% -----------------------------------------------------------------------------
% Motivation slide
\begin{frame}[t]{Motivation}
	\begin{itemize}
	    \item Millimeter wave indoor localization system based on active radar sensing of local passive reference points \cite{schlachter_indoor_2024}
            \begin{figure}
            	\centering
            	\includegraphics[scale=.28]{figures/thesis_plot.png}
            	\caption{The conceptual view of the millimeter wave indoor localization system}
            \end{figure}
        \item Ray tracing channel simulation
        \item Large and realistic environment
	\end{itemize}
\end{frame}



\begin{frame}[t]{Motivation}
	\begin{itemize}
	    \item Frequency-modulated continuous wave (FMCW) radar
        \vspace{0.5\baselineskip}
            \begin{figure}
            	\centering
            	\includegraphics[scale=.3]{figures/FMCW_principle.png}
            	\caption{The principle of the FMCW radar}
            \end{figure}
	\end{itemize}
\end{frame}



% Goal slide
\begin{frame}[t]{Goal}
	\begin{itemize}
	    \item The signal flow
            \begin{figure}
            	\centering
            	\includegraphics[scale=.3]{figures/signal_flow_pre.png}
            	\caption{The view of the signal flow \cite{hafner_parameter_2021}}
            \end{figure}
        \item Range-Doppler (RD) map and constant false alarm rate (CFAR) detection
	\end{itemize}
\end{frame}

% -----------------------------------------------------------------------------
% This is the table of contents. You can insert a motivation before or after this slide.
\begin{frame}
	\ifthenelse{\equal{\lang}{ngerman}}{
		\frametitle{Inhaltsverzeichnis}
	}{
		\frametitle{Table of Contents}
	}
	\tableofcontents
\end{frame}

% Add an extra slide at the beginning of each section while highlighting the current section
% Use \section* to skip the slide once or comment the following to skip all overview slides.
\AtBeginSection[]
{
	\begin{frame}<beamer>
		\ifthenelse{\equal{\lang}{ngerman}}{
			\frametitle{Inhaltsverzeichnis}
		}{
			\frametitle{Table of Contents}
		}
% 		\frametitle{\contentsname}
		\tableofcontents[currentsection]
	\end{frame}
}

%% =========
\section{Modeling of the channel simulation}
\subsection{Modeling of the room and raytracing}
\setcounter{section}{1}
\setcounter{figure}{0}
% -----------------------------------------------------------------------------
\begin{frame}[t]{Blender}

\begin{itemize}
	\item Modeling of the reflectors
    \vspace{1.0\baselineskip}
    \begin{figure}
        \centering
        \begin{minipage}{0.45\textwidth}
            \centering
            \includegraphics[height=0.5\textheight]{figures/tri_reflector.png}
            \caption{Trihedral corner reflector}
        \end{minipage}
        \begin{minipage}{0.45\textwidth}
            \centering
            \includegraphics[height=0.5\textheight]{figures/octa_reflector.png}
            \caption{Octahedral reflector}
        \end{minipage}
    \end{figure}
\end{itemize}
\end{frame}


\begin{frame}[t]{Blender}

\begin{itemize}
	\item Modeling of the room model
    \vspace{1.0\baselineskip}
    \begin{figure}
        \centering
        \begin{minipage}{0.45\textwidth}
            \centering
            \includegraphics[height=0.7\textwidth]{figures/empty_room_without_ceiling.png}
            \caption{Room model with reflectors}
        \end{minipage}
        \begin{minipage}{0.45\textwidth}
            \centering
            \includegraphics[height=0.7\textwidth]{figures/furniture_simple.png}
            \caption{Room model with furniture \cite{blender_furniture}}
        \end{minipage}
    
    \end{figure}
\end{itemize}
\end{frame}



\begin{frame}[t]{Raytracer}
	\begin{itemize}
        \item Setup of the raytracing (MaxNumreflections, MaxAbsolutePathLoss, Method, etc.)
	    \item The shooting and bouncing ray (SBR) method
        \vspace{0.5\baselineskip}
            \begin{figure}
            	\centering
            	\includegraphics[scale=.5]{figures/ray_tracing_sbr_method.png}
            	\caption{The view of the SBR method \cite{ray_tracer_inputs}}
            \end{figure}
	\end{itemize}
\end{frame}





\begin{frame}[t]{Raytracing}
	\begin{itemize}
	    \item The view of raytracing in the empty room model and room model with furniture
        \vspace{0.5\baselineskip}
            \begin{figure}
                \centering
                \begin{minipage}{0.45\textwidth}
                    \centering
                    \includegraphics[height=0.7\textwidth]{figures/empty_room_simulation_right.png}
                    \caption{In empty room model}
                \end{minipage}
                \begin{minipage}{0.45\textwidth}
                    \centering
                    \includegraphics[height=0.7\textwidth]{figures/furniture_simulation_right.png}
                    \caption{In room with furniture}
                \end{minipage}
            \end{figure}
	\end{itemize}
\end{frame}



\begin{frame}[t]{Raytracer}
	\begin{itemize}
	    \item The view of raytracing in the room model with reflectors and inside the reflector
        \vspace{0.5\baselineskip}
            \begin{figure}
                \centering
                \begin{minipage}{0.45\textwidth}
                    \centering
                    \includegraphics[height=0.7\textwidth]{figures/reflector_simulation_right.png}
                    \caption{In room with reflectors}
                \end{minipage}
                \begin{minipage}{0.45\textwidth}
                    \centering
                    \includegraphics[height=0.7\textwidth]{figures/reflections_inside_reflector.png}
                    \caption{The view inside the reflector}
                \end{minipage}
            \end{figure}
	\end{itemize}
\end{frame}


\subsection{LOS rays between the reflectors and radar}
% -----------------------------------------------------------------------------
\begin{frame}[t]{Line-of-sight (LOS) rays}
    \begin{itemize}
	    \item Setting the receivers at the positions of reflectors to obtain LOS rays
        \vspace{0.5\baselineskip}
            \begin{figure}
                \centering
                \begin{minipage}{0.45\textwidth}
                    \centering
                    \includegraphics[height=0.7\textwidth]{figures/LOS_5_reflectors.png}
                    \caption{LOS rays from five trihedral corner reflectors}
                \end{minipage}
                \begin{minipage}{0.45\textwidth}
                    \centering
                    \includegraphics[height=0.7\textwidth]{figures/LOS_more_reflectors.png}
                    \caption{LOS rays from multiple octahedral reflectors}
                \end{minipage}
            \end{figure}
	\end{itemize}
\end{frame}



\begin{frame}[t]{Path loss (PL)}
    \begin{equation}
        \centering
        PL = 10\cdot\log_{10}\left(\frac{G_{T}G_{R}\lambda^{2}\cdot\sigma}{(4\pi)^{3} r^{4}}\right) \cite{richards_principles_2010},
        \label{path loss formula}
    \end{equation}
    where
    \begin{itemize}
        \item $G_T$ and $G_R$ are the gain values of transmitter and receiver respectively.
        \item $\lambda$ is the wavelength of the ray.
        \item $\sigma$ is the radar cross section.
        \item $r$ represents the propagation distance of LOS rays.
    \end{itemize}
\end{frame}




\subsection{Signal processing}
% -----------------------------------------------------------------------------
\begin{frame}[t]{The signal model}
    \begin{equation}
        \centering
        x(n_s, n_p) = \sum_{m} A_m \exp \left( 2\pi j \left(\frac{2 B r_{m}}{T_c c} T_s n_s + \frac{2 f v_{m}}{c} T_p n_p \right) \right) + n(n_s, n_p)\,\mathrm{\cite{ouza_simple_2017}},
    \end{equation}
    where
    \begin{itemize}
        \item $x(n_s, n_p)$ is an element of a 2D matrix $\mathbf{X}$, i.e. baseband signal,
        \item $n_s$ = 1...256 is the fast time samples,
        \item $n_p$=1...32 is the slow time samples,
        \item $m$ represents each ray in the channel simulation,
        \item $A_m = \sqrt{P_T \cdot 10^{-PL_m / 10} \cdot R}$,
        \item the first term for the delay and the second term for the Doppler effect,
        \item $n(n_s, n_p)$ is the Gaussian noise $n \sim \mathcal{CN}(0, V_{noise})$.
    \end{itemize}
\end{frame}


\begin{frame}[t]{Range-Doppler map}
    \begin{itemize}
        \item The view of 2D fast Fourier transform (FFT) and range-Doppler map
         \vspace{0.5\baselineskip}
            \begin{figure}
                \centering
                \begin{minipage}{0.45\textwidth}
                    \centering
                    \includegraphics[height=0.8\textwidth]{figures/2D_FFT.png}
                    \caption{The view of 2D FFT \cite{kronauge_new_2014}}
                \end{minipage}
                \begin{minipage}{0.45\textwidth}
                    \centering
                    \includegraphics[height=0.8\textwidth]{figures/2r_furniture.png}
                    \caption{Range-Doppler map}
                \end{minipage}
            \end{figure}
    \end{itemize}
\end{frame}



\begin{frame}[t]{Constant false alarm rate (CFAR) detection}
	\begin{itemize}
        \item The CFAR plots for the empty room model and room model with reflectors
         \vspace{0.5\baselineskip}
            \begin{figure}
                \centering
                \begin{minipage}{0.45\textwidth}
                    \centering
                    \includegraphics[height=0.8\textwidth]{figures/2c_empty_pre.png}
                    \caption{In the empty room}
                \end{minipage}
                \begin{minipage}{0.45\textwidth}
                    \centering
                    \includegraphics[height=0.8\textwidth]{figures/2c_reflectors_pre.png}
                    \caption{In the room with reflectors}
                \end{minipage}
            \end{figure}
    \end{itemize}
\end{frame}




\begin{frame}[t]{Constant false alarm rate (CFAR) detection}
	\begin{itemize}
        \item The CFAR plots for the room model with reflectors and with furniture
         \vspace{0.5\baselineskip}
            \begin{figure}
                \centering
                \begin{minipage}{0.45\textwidth}
                    \centering
                    \includegraphics[height=0.8\textwidth]{figures/2c_reflectors_pre.png}
                    \caption{In the room with reflectors}
                \end{minipage}
                \begin{minipage}{0.45\textwidth}
                    \centering
                    \includegraphics[height=0.8\textwidth]{figures/2c_furniture_pre.png}
                    \caption{In the room with furniture}
                \end{minipage}
            \end{figure}
    \end{itemize}
\end{frame}




\section{Antenna radiation pattern}
\subsection{Antenna designer}
\setcounter{section}{2}
\setcounter{figure}{0}
% -----------------------------------------------------------------------------

\begin{frame}[t]{Antenna designer}
	\begin{itemize}
	    \item "Optimization" function to obtain maximum gain value
        \vspace{1.0\baselineskip}
            \begin{figure}
            	\centering
            	\includegraphics[scale=.5]{figures/max_gain.png}
            	\caption{The view of the maximum gain}
            \end{figure}
	\end{itemize}
\end{frame}




\begin{frame}[t]{Antenna designer}
	\begin{itemize}
	    \item The gain as $az=0^{\circ}$ between the first version and after tuning
        \vspace{1.0\baselineskip}
            \begin{figure}
                \centering
                \begin{minipage}{0.45\textwidth}
                    \centering
                    \includegraphics[height=0.55\textwidth]{figures/antenna_gain_optimization.png}
                    \caption{The gain of the first version}
                \end{minipage}
                \begin{minipage}{0.45\textwidth}
                    \centering
                    \includegraphics[height=0.55\textwidth]{figures/antenna_gain_latest.png}
                    \caption{The gain after tuning}
                \end{minipage}
            \end{figure}
	\end{itemize}
\end{frame}




\begin{frame}[t]{Antenna designer}
	\begin{itemize}
	    \item The view and radiation pattern of the final antenna
        \vspace{1.5\baselineskip}
            \begin{figure}
                \centering
                \begin{minipage}{0.45\textwidth}
                    \centering
                    \includegraphics[height=0.6\textwidth]{figures/antenna_plot_latest.png}
                    \caption{View of the final antenna}
                \end{minipage}
                \begin{minipage}{0.45\textwidth}
                    \centering
                    \includegraphics[height=0.6\textwidth]{figures/radiation_pattern_latest.png}
                    \caption{3D radiation pattern}
                \end{minipage}
            \end{figure}
	\end{itemize}
\end{frame}



\subsection{Results}
% -----------------------------------------------------------------------------
\begin{frame}[t]{Results}
	\begin{itemize}
	    \item The constant false alarm rate (CFAR) plots about the radiation pattern
        \vspace{0.5\baselineskip}
            \begin{figure}
                \centering
                \begin{minipage}{0.45\textwidth}
                    \centering
                    \includegraphics[height=0.8\textwidth]{figures/1c_empty.png}
                    \caption{CFAR without radiation pattern}
                \end{minipage}
                \begin{minipage}{0.45\textwidth}
                    \centering
                    \includegraphics[height=0.8\textwidth]{figures/3c_beampattern.png}
                    \caption{CFAR with radiation pattern}
                \end{minipage}
            \end{figure}
	\end{itemize}
\end{frame}




\begin{frame}[t]{Results}
	\begin{itemize}
	    \item The view of the asymmetric radiation pattern in the room
        \vspace{1.0\baselineskip}
            \begin{figure}
            	\centering
            	\includegraphics[scale=.17]{figures/asymmetric_antenna.png}
            	\caption{The view of the asymmetric pattern}
            \end{figure}
	\end{itemize}
\end{frame}


\section{Reflector radar cross section}
\subsection{Radar cross section}
\setcounter{section}{3}
\setcounter{figure}{0}
% -----------------------------------------------------------------------------
\begin{frame}[t]{Radar cross section (RCS)}
	\begin{itemize}
	    \item View of the RCS for the tetrahedral object with default parameters
        \vspace{0.5\baselineskip}
            \begin{figure}
                \centering
                \begin{minipage}{0.47\textwidth}
                    \centering
                    \includegraphics[height=0.7\textwidth]{figures/mesh_tetrahedral.png}
                    \caption{Mesh of the tetrahedral object}
                \end{minipage}
                \begin{minipage}{0.47\textwidth}
                    \centering
                    \includegraphics[height=0.7\textwidth]{figures/rcs_tetrahedral.png}
                    \caption{2D RCS of the tetrahedral object}
                \end{minipage}
            \end{figure}
	\end{itemize}
\end{frame}





\begin{frame}[t]{Radar cross section (RCS)}
	\begin{itemize}
	    \item View of the RCS for the tetrahedral object with scaled parameters
        \vspace{0.5\baselineskip}
            \begin{figure}
                \centering
                \begin{minipage}{0.47\textwidth}
                    \centering
                    \includegraphics[height=0.7\textwidth]{figures/rcs_tetrahedral.png}
                    \caption{2D RCS without scaling}
                \end{minipage}
                \begin{minipage}{0.47\textwidth}
                    \centering
                    \includegraphics[height=0.7\textwidth]{figures/rcs_tetrahedral_scaled.png}
                    \caption{2D RCS with scaling}
                \end{minipage}
            \end{figure}
	\end{itemize}
\end{frame}





\begin{frame}[t]{Radar cross section (RCS)}
	\begin{itemize}
	    \item View of the RCS for the trihedral corner reflector
        \vspace{0.5\baselineskip}
            \begin{figure}
                \centering
                \begin{minipage}{0.47\textwidth}
                    \centering
                    \includegraphics[height=0.7\textwidth]{figures/mesh_cr.png}
                    \caption{Mesh of trihedral corner reflector}
                \end{minipage}
                \begin{minipage}{0.47\textwidth}
                    \centering
                    \includegraphics[height=0.7\textwidth]{figures/rcs_cr.png}
                    \caption{2D RCS of trihedral reflector}
                \end{minipage}
            \end{figure}
	\end{itemize}
\end{frame}




\begin{frame}[t]{Radar cross section (RCS)}
	\begin{itemize}
	    \item 3D RCS views for both reflectors
        \vspace{0.5\baselineskip}
            \begin{figure}
                \centering
                \begin{minipage}{0.47\textwidth}
                    \centering
                    \includegraphics[height=0.7\textwidth]{figures/trihe_3d_rcs.png}
                    \caption{3D RCS of trihedral reflector}
                \end{minipage}
                \begin{minipage}{0.47\textwidth}
                    \centering
                    \includegraphics[height=0.7\textwidth]{figures/octa_3d_rcs.png}
                    \caption{3D RCS of octahedral reflector}
                \end{minipage}
            \end{figure}
	\end{itemize}
\end{frame}





\subsection{Results}
% -----------------------------------------------------------------------------
\begin{frame}[t]{Results}
	\begin{itemize}
	    \item The CFAR plots about the RCS pattern
        \vspace{0.5\baselineskip}
            \begin{figure}
                \centering
                \begin{minipage}{0.45\textwidth}
                    \centering
                    \includegraphics[height=0.8\textwidth]{figures/3c_beampattern.png}
                    \caption{With isotropic RCS pattern}
                \end{minipage}
                \begin{minipage}{0.45\textwidth}
                    \centering
                    \includegraphics[height=0.8\textwidth]{figures/4c_empty.png}
                    \caption{With anisotropic RCS pattern}
                \end{minipage}
            \end{figure}
	\end{itemize}
\end{frame}




\section{Motion of the robot}
\subsection{Trajectory generation}
\setcounter{section}{4}
\setcounter{figure}{0}
% -----------------------------------------------------------------------------
\begin{frame}[t]{Trajectory generation}
	\begin{itemize}
	    \item The trajectory generation for the robot along the wall and furniture
        \vspace{0.5\baselineskip}
            \begin{figure}
            	\centering
            	\includegraphics[scale=.18]{figures/trajectory.png}
            	\caption{The trajectory of the robot}
            \end{figure}
	\end{itemize}
\end{frame}





\subsection{Orientation}
% -----------------------------------------------------------------------------
\begin{frame}[t]{Orientation}
	\begin{itemize}
	    \item The orientation of the antenna during motion
        \vspace{1.0\baselineskip}
            \begin{figure}
                \centering
                \begin{minipage}{0.47\textwidth}
                    \centering
                    \includegraphics[height=0.7\textwidth]{figures/original_pattern.png}
                    \caption{Radiation pattern before rotation}
                \end{minipage}
                \begin{minipage}{0.47\textwidth}
                    \centering
                    \includegraphics[height=0.7\textwidth]{figures/orientated_pattern.png}
                    \caption{Radiation pattern after rotation}
                \end{minipage}
            \end{figure}
	\end{itemize}
\end{frame}





\section{Evaluation}
\setcounter{section}{5}
\setcounter{figure}{0}
% -----------------------------------------------------------------------------

\begin{frame}[t]{Evaluation}
	\begin{itemize}
	    \item The CFAR plots difference in units watt (W) and decibel watt (dBW)
        \vspace{0.5\baselineskip}
            \begin{figure}
                \centering
                \begin{minipage}{0.45\textwidth}
                    \centering
                    \includegraphics[height=0.8\textwidth]{figures/1c_empty.png}
                    \caption{CFAR plot in unit W}
                \end{minipage}
                \begin{minipage}{0.45\textwidth}
                    \centering
                    \includegraphics[height=0.8\textwidth]{figures/1c_empty_dBW.png}
                    \caption{CFAR plot in unit dBW}
                \end{minipage}
            \end{figure}
	\end{itemize}
\end{frame}





\begin{frame}[t]{Evaluation}
	\begin{itemize}
	    \item The CFAR plots with various positions and amount of reflectors in unit W
        \vspace{0.5\baselineskip}
            \begin{figure}
                \centering
                \begin{minipage}{0.45\textwidth}
                    \centering
                    \includegraphics[height=0.8\textwidth]{figures/4c_empty.png}
                    \caption{CFAR plots with five trihedral corner reflectors}
                \end{minipage}
                \begin{minipage}{0.45\textwidth}
                    \centering
                    \includegraphics[height=0.8\textwidth]{figures/5c_octahedral.png}
                    \caption{CFAR plots with multiple octahedral reflectors}
                \end{minipage}
            \end{figure}
	\end{itemize}
\end{frame}





\begin{frame}[t]{Evaluation}
	\begin{itemize}
	    \item The meaning of the unit in the case of the multiple reflectors
        \vspace{0.5\baselineskip}
            \begin{figure}
                \centering
                \begin{minipage}{0.45\textwidth}
                    \centering
                    \includegraphics[height=0.8\textwidth]{figures/5c_octahedral.png}
                    \caption{CFAR plots in the unit W}
                \end{minipage}
                \begin{minipage}{0.45\textwidth}
                    \centering
                    \includegraphics[height=0.8\textwidth]{figures/5c_octahedral_dBW.png}
                    \caption{CFAR plots in the unit dBW}
                \end{minipage}
            \end{figure}
	\end{itemize}
\end{frame}





% \begin{frame}[t]{Evaluation}
% 	\begin{itemize}
% 	    \item The CFAR plots with different maximum number of reflections
%         \vspace{0.5\baselineskip}
%             \begin{figure}
%                 \centering
%                 \begin{minipage}{0.45\textwidth}
%                     \centering
%                     \includegraphics[height=0.8\textwidth]{figures/4c_rcs.png}
%                     \caption{Max four reflections}
%                 \end{minipage}
%                 \begin{minipage}{0.45\textwidth}
%                     \centering
%                     \includegraphics[height=0.8\textwidth]{figures/8c_reflection.png}
%                     \caption{Max two reflections}
%                 \end{minipage}
%             \end{figure}
% 	\end{itemize}
% \end{frame}




\begin{frame}[t]{Evaluation}
	\begin{itemize}
	    \item The CFAR plots along the trajectory generated in GIF format
        \vspace{0.5\baselineskip}
	\end{itemize}
    \centering
    \begin{figure}
        \centering
        \animategraphics[width=0.5\linewidth, autoplay=True]{2}{cfar_png/}{1}{54}
        \caption{CFAR plots along the trajectory}
    \end{figure}
\end{frame}




\section{Summary and outlook}
% -----------------------------------------------------------------------------
\begin{frame}[t]{Summary and outlook}
	\begin{itemize}
	    \item Summary: The ray tracing channel model works well and meets the goal.
        \vspace{0.5\baselineskip}
        \item Outlook:
        \vspace{0.5\baselineskip}
        \begin{itemize}
            \item Ray tracing with the surface diffraction
            \vspace{0.5\baselineskip}
            \item Antenna designer
            \vspace{0.5\baselineskip}
            \item RCS pattern simulation without scaling
            \vspace{0.5\baselineskip}
            \item Parallelizable computation of the simulation on GPU
            \vspace{0.5\baselineskip}
            \item Optimization of the number and arrangement of the reflectors
            \vspace{0.5\baselineskip}
            \item Combining with the deep learning for indoor localization system
        \end{itemize}
        
	\end{itemize}
\end{frame}



% -----------------------------------------------------------------------------
% Reference slide
\renewcommand*{\bibfont}{\scriptsize}
\begin{frame}[t]{Reference}
	\ifthenelse{\equal{\doclang}{german}}{
    	\bibliographystyle{IEEEtran_ISSger}
    }{
    	\bibliographystyle{IEEEtran_ISS}
    }
    \printbibliography{}
\end{frame}




% -----------------------------------------------------------------------------
\begin{frame}[t]
    \centering
    \fontsize{30}{120}\selectfont
	\textbf{Thank you for your attention}
\end{frame}

% -----------------------------------------------------------------------------


\end{document}
